
% Default to the notebook output style

    


% Inherit from the specified cell style.




    
\documentclass[11pt]{article}

    
    
    \usepackage[T1]{fontenc}
    % Nicer default font (+ math font) than Computer Modern for most use cases
    \usepackage{mathpazo}

    % Basic figure setup, for now with no caption control since it's done
    % automatically by Pandoc (which extracts ![](path) syntax from Markdown).
    \usepackage{graphicx}
    % We will generate all images so they have a width \maxwidth. This means
    % that they will get their normal width if they fit onto the page, but
    % are scaled down if they would overflow the margins.
    \makeatletter
    \def\maxwidth{\ifdim\Gin@nat@width>\linewidth\linewidth
    \else\Gin@nat@width\fi}
    \makeatother
    \let\Oldincludegraphics\includegraphics
    % Set max figure width to be 80% of text width, for now hardcoded.
    \renewcommand{\includegraphics}[1]{\Oldincludegraphics[width=.8\maxwidth]{#1}}
    % Ensure that by default, figures have no caption (until we provide a
    % proper Figure object with a Caption API and a way to capture that
    % in the conversion process - todo).
    \usepackage{caption}
    \DeclareCaptionLabelFormat{nolabel}{}
    \captionsetup{labelformat=nolabel}

    \usepackage{adjustbox} % Used to constrain images to a maximum size 
    \usepackage{xcolor} % Allow colors to be defined
    \usepackage{enumerate} % Needed for markdown enumerations to work
    \usepackage{geometry} % Used to adjust the document margins
    \usepackage{amsmath} % Equations
    \usepackage{amssymb} % Equations
    \usepackage{textcomp} % defines textquotesingle
    % Hack from http://tex.stackexchange.com/a/47451/13684:
    \AtBeginDocument{%
        \def\PYZsq{\textquotesingle}% Upright quotes in Pygmentized code
    }
    \usepackage{upquote} % Upright quotes for verbatim code
    \usepackage{eurosym} % defines \euro
    \usepackage[mathletters]{ucs} % Extended unicode (utf-8) support
    \usepackage[utf8x]{inputenc} % Allow utf-8 characters in the tex document
    \usepackage{fancyvrb} % verbatim replacement that allows latex
    \usepackage{grffile} % extends the file name processing of package graphics 
                         % to support a larger range 
    % The hyperref package gives us a pdf with properly built
    % internal navigation ('pdf bookmarks' for the table of contents,
    % internal cross-reference links, web links for URLs, etc.)
    \usepackage{hyperref}
    \usepackage{longtable} % longtable support required by pandoc >1.10
    \usepackage{booktabs}  % table support for pandoc > 1.12.2
    \usepackage[inline]{enumitem} % IRkernel/repr support (it uses the enumerate* environment)
    \usepackage[normalem]{ulem} % ulem is needed to support strikethroughs (\sout)
                                % normalem makes italics be italics, not underlines
    \usepackage{mathrsfs}
    

    
    
    % Colors for the hyperref package
    \definecolor{urlcolor}{rgb}{0,.145,.698}
    \definecolor{linkcolor}{rgb}{.71,0.21,0.01}
    \definecolor{citecolor}{rgb}{.12,.54,.11}

    % ANSI colors
    \definecolor{ansi-black}{HTML}{3E424D}
    \definecolor{ansi-black-intense}{HTML}{282C36}
    \definecolor{ansi-red}{HTML}{E75C58}
    \definecolor{ansi-red-intense}{HTML}{B22B31}
    \definecolor{ansi-green}{HTML}{00A250}
    \definecolor{ansi-green-intense}{HTML}{007427}
    \definecolor{ansi-yellow}{HTML}{DDB62B}
    \definecolor{ansi-yellow-intense}{HTML}{B27D12}
    \definecolor{ansi-blue}{HTML}{208FFB}
    \definecolor{ansi-blue-intense}{HTML}{0065CA}
    \definecolor{ansi-magenta}{HTML}{D160C4}
    \definecolor{ansi-magenta-intense}{HTML}{A03196}
    \definecolor{ansi-cyan}{HTML}{60C6C8}
    \definecolor{ansi-cyan-intense}{HTML}{258F8F}
    \definecolor{ansi-white}{HTML}{C5C1B4}
    \definecolor{ansi-white-intense}{HTML}{A1A6B2}
    \definecolor{ansi-default-inverse-fg}{HTML}{FFFFFF}
    \definecolor{ansi-default-inverse-bg}{HTML}{000000}

    % commands and environments needed by pandoc snippets
    % extracted from the output of `pandoc -s`
    \providecommand{\tightlist}{%
      \setlength{\itemsep}{0pt}\setlength{\parskip}{0pt}}
    \DefineVerbatimEnvironment{Highlighting}{Verbatim}{commandchars=\\\{\}}
    % Add ',fontsize=\small' for more characters per line
    \newenvironment{Shaded}{}{}
    \newcommand{\KeywordTok}[1]{\textcolor[rgb]{0.00,0.44,0.13}{\textbf{{#1}}}}
    \newcommand{\DataTypeTok}[1]{\textcolor[rgb]{0.56,0.13,0.00}{{#1}}}
    \newcommand{\DecValTok}[1]{\textcolor[rgb]{0.25,0.63,0.44}{{#1}}}
    \newcommand{\BaseNTok}[1]{\textcolor[rgb]{0.25,0.63,0.44}{{#1}}}
    \newcommand{\FloatTok}[1]{\textcolor[rgb]{0.25,0.63,0.44}{{#1}}}
    \newcommand{\CharTok}[1]{\textcolor[rgb]{0.25,0.44,0.63}{{#1}}}
    \newcommand{\StringTok}[1]{\textcolor[rgb]{0.25,0.44,0.63}{{#1}}}
    \newcommand{\CommentTok}[1]{\textcolor[rgb]{0.38,0.63,0.69}{\textit{{#1}}}}
    \newcommand{\OtherTok}[1]{\textcolor[rgb]{0.00,0.44,0.13}{{#1}}}
    \newcommand{\AlertTok}[1]{\textcolor[rgb]{1.00,0.00,0.00}{\textbf{{#1}}}}
    \newcommand{\FunctionTok}[1]{\textcolor[rgb]{0.02,0.16,0.49}{{#1}}}
    \newcommand{\RegionMarkerTok}[1]{{#1}}
    \newcommand{\ErrorTok}[1]{\textcolor[rgb]{1.00,0.00,0.00}{\textbf{{#1}}}}
    \newcommand{\NormalTok}[1]{{#1}}
    
    % Additional commands for more recent versions of Pandoc
    \newcommand{\ConstantTok}[1]{\textcolor[rgb]{0.53,0.00,0.00}{{#1}}}
    \newcommand{\SpecialCharTok}[1]{\textcolor[rgb]{0.25,0.44,0.63}{{#1}}}
    \newcommand{\VerbatimStringTok}[1]{\textcolor[rgb]{0.25,0.44,0.63}{{#1}}}
    \newcommand{\SpecialStringTok}[1]{\textcolor[rgb]{0.73,0.40,0.53}{{#1}}}
    \newcommand{\ImportTok}[1]{{#1}}
    \newcommand{\DocumentationTok}[1]{\textcolor[rgb]{0.73,0.13,0.13}{\textit{{#1}}}}
    \newcommand{\AnnotationTok}[1]{\textcolor[rgb]{0.38,0.63,0.69}{\textbf{\textit{{#1}}}}}
    \newcommand{\CommentVarTok}[1]{\textcolor[rgb]{0.38,0.63,0.69}{\textbf{\textit{{#1}}}}}
    \newcommand{\VariableTok}[1]{\textcolor[rgb]{0.10,0.09,0.49}{{#1}}}
    \newcommand{\ControlFlowTok}[1]{\textcolor[rgb]{0.00,0.44,0.13}{\textbf{{#1}}}}
    \newcommand{\OperatorTok}[1]{\textcolor[rgb]{0.40,0.40,0.40}{{#1}}}
    \newcommand{\BuiltInTok}[1]{{#1}}
    \newcommand{\ExtensionTok}[1]{{#1}}
    \newcommand{\PreprocessorTok}[1]{\textcolor[rgb]{0.74,0.48,0.00}{{#1}}}
    \newcommand{\AttributeTok}[1]{\textcolor[rgb]{0.49,0.56,0.16}{{#1}}}
    \newcommand{\InformationTok}[1]{\textcolor[rgb]{0.38,0.63,0.69}{\textbf{\textit{{#1}}}}}
    \newcommand{\WarningTok}[1]{\textcolor[rgb]{0.38,0.63,0.69}{\textbf{\textit{{#1}}}}}
    
    
    % Define a nice break command that doesn't care if a line doesn't already
    % exist.
    \def\br{\hspace*{\fill} \\* }
    % Math Jax compatibility definitions
    \def\gt{>}
    \def\lt{<}
    \let\Oldtex\TeX
    \let\Oldlatex\LaTeX
    \renewcommand{\TeX}{\textrm{\Oldtex}}
    \renewcommand{\LaTeX}{\textrm{\Oldlatex}}
    % Document parameters
    % Document title
    \title{Machine Learning Engineer Nanodegree \\ New York City Taxi Trip Duration}
    \date{\today}
    \author{Cosa Santos}
    
    
    
    
    

    % Pygments definitions
    
\makeatletter
\def\PY@reset{\let\PY@it=\relax \let\PY@bf=\relax%
    \let\PY@ul=\relax \let\PY@tc=\relax%
    \let\PY@bc=\relax \let\PY@ff=\relax}
\def\PY@tok#1{\csname PY@tok@#1\endcsname}
\def\PY@toks#1+{\ifx\relax#1\empty\else%
    \PY@tok{#1}\expandafter\PY@toks\fi}
\def\PY@do#1{\PY@bc{\PY@tc{\PY@ul{%
    \PY@it{\PY@bf{\PY@ff{#1}}}}}}}
\def\PY#1#2{\PY@reset\PY@toks#1+\relax+\PY@do{#2}}

\expandafter\def\csname PY@tok@w\endcsname{\def\PY@tc##1{\textcolor[rgb]{0.73,0.73,0.73}{##1}}}
\expandafter\def\csname PY@tok@c\endcsname{\let\PY@it=\textit\def\PY@tc##1{\textcolor[rgb]{0.25,0.50,0.50}{##1}}}
\expandafter\def\csname PY@tok@cp\endcsname{\def\PY@tc##1{\textcolor[rgb]{0.74,0.48,0.00}{##1}}}
\expandafter\def\csname PY@tok@k\endcsname{\let\PY@bf=\textbf\def\PY@tc##1{\textcolor[rgb]{0.00,0.50,0.00}{##1}}}
\expandafter\def\csname PY@tok@kp\endcsname{\def\PY@tc##1{\textcolor[rgb]{0.00,0.50,0.00}{##1}}}
\expandafter\def\csname PY@tok@kt\endcsname{\def\PY@tc##1{\textcolor[rgb]{0.69,0.00,0.25}{##1}}}
\expandafter\def\csname PY@tok@o\endcsname{\def\PY@tc##1{\textcolor[rgb]{0.40,0.40,0.40}{##1}}}
\expandafter\def\csname PY@tok@ow\endcsname{\let\PY@bf=\textbf\def\PY@tc##1{\textcolor[rgb]{0.67,0.13,1.00}{##1}}}
\expandafter\def\csname PY@tok@nb\endcsname{\def\PY@tc##1{\textcolor[rgb]{0.00,0.50,0.00}{##1}}}
\expandafter\def\csname PY@tok@nf\endcsname{\def\PY@tc##1{\textcolor[rgb]{0.00,0.00,1.00}{##1}}}
\expandafter\def\csname PY@tok@nc\endcsname{\let\PY@bf=\textbf\def\PY@tc##1{\textcolor[rgb]{0.00,0.00,1.00}{##1}}}
\expandafter\def\csname PY@tok@nn\endcsname{\let\PY@bf=\textbf\def\PY@tc##1{\textcolor[rgb]{0.00,0.00,1.00}{##1}}}
\expandafter\def\csname PY@tok@ne\endcsname{\let\PY@bf=\textbf\def\PY@tc##1{\textcolor[rgb]{0.82,0.25,0.23}{##1}}}
\expandafter\def\csname PY@tok@nv\endcsname{\def\PY@tc##1{\textcolor[rgb]{0.10,0.09,0.49}{##1}}}
\expandafter\def\csname PY@tok@no\endcsname{\def\PY@tc##1{\textcolor[rgb]{0.53,0.00,0.00}{##1}}}
\expandafter\def\csname PY@tok@nl\endcsname{\def\PY@tc##1{\textcolor[rgb]{0.63,0.63,0.00}{##1}}}
\expandafter\def\csname PY@tok@ni\endcsname{\let\PY@bf=\textbf\def\PY@tc##1{\textcolor[rgb]{0.60,0.60,0.60}{##1}}}
\expandafter\def\csname PY@tok@na\endcsname{\def\PY@tc##1{\textcolor[rgb]{0.49,0.56,0.16}{##1}}}
\expandafter\def\csname PY@tok@nt\endcsname{\let\PY@bf=\textbf\def\PY@tc##1{\textcolor[rgb]{0.00,0.50,0.00}{##1}}}
\expandafter\def\csname PY@tok@nd\endcsname{\def\PY@tc##1{\textcolor[rgb]{0.67,0.13,1.00}{##1}}}
\expandafter\def\csname PY@tok@s\endcsname{\def\PY@tc##1{\textcolor[rgb]{0.73,0.13,0.13}{##1}}}
\expandafter\def\csname PY@tok@sd\endcsname{\let\PY@it=\textit\def\PY@tc##1{\textcolor[rgb]{0.73,0.13,0.13}{##1}}}
\expandafter\def\csname PY@tok@si\endcsname{\let\PY@bf=\textbf\def\PY@tc##1{\textcolor[rgb]{0.73,0.40,0.53}{##1}}}
\expandafter\def\csname PY@tok@se\endcsname{\let\PY@bf=\textbf\def\PY@tc##1{\textcolor[rgb]{0.73,0.40,0.13}{##1}}}
\expandafter\def\csname PY@tok@sr\endcsname{\def\PY@tc##1{\textcolor[rgb]{0.73,0.40,0.53}{##1}}}
\expandafter\def\csname PY@tok@ss\endcsname{\def\PY@tc##1{\textcolor[rgb]{0.10,0.09,0.49}{##1}}}
\expandafter\def\csname PY@tok@sx\endcsname{\def\PY@tc##1{\textcolor[rgb]{0.00,0.50,0.00}{##1}}}
\expandafter\def\csname PY@tok@m\endcsname{\def\PY@tc##1{\textcolor[rgb]{0.40,0.40,0.40}{##1}}}
\expandafter\def\csname PY@tok@gh\endcsname{\let\PY@bf=\textbf\def\PY@tc##1{\textcolor[rgb]{0.00,0.00,0.50}{##1}}}
\expandafter\def\csname PY@tok@gu\endcsname{\let\PY@bf=\textbf\def\PY@tc##1{\textcolor[rgb]{0.50,0.00,0.50}{##1}}}
\expandafter\def\csname PY@tok@gd\endcsname{\def\PY@tc##1{\textcolor[rgb]{0.63,0.00,0.00}{##1}}}
\expandafter\def\csname PY@tok@gi\endcsname{\def\PY@tc##1{\textcolor[rgb]{0.00,0.63,0.00}{##1}}}
\expandafter\def\csname PY@tok@gr\endcsname{\def\PY@tc##1{\textcolor[rgb]{1.00,0.00,0.00}{##1}}}
\expandafter\def\csname PY@tok@ge\endcsname{\let\PY@it=\textit}
\expandafter\def\csname PY@tok@gs\endcsname{\let\PY@bf=\textbf}
\expandafter\def\csname PY@tok@gp\endcsname{\let\PY@bf=\textbf\def\PY@tc##1{\textcolor[rgb]{0.00,0.00,0.50}{##1}}}
\expandafter\def\csname PY@tok@go\endcsname{\def\PY@tc##1{\textcolor[rgb]{0.53,0.53,0.53}{##1}}}
\expandafter\def\csname PY@tok@gt\endcsname{\def\PY@tc##1{\textcolor[rgb]{0.00,0.27,0.87}{##1}}}
\expandafter\def\csname PY@tok@err\endcsname{\def\PY@bc##1{\setlength{\fboxsep}{0pt}\fcolorbox[rgb]{1.00,0.00,0.00}{1,1,1}{\strut ##1}}}
\expandafter\def\csname PY@tok@kc\endcsname{\let\PY@bf=\textbf\def\PY@tc##1{\textcolor[rgb]{0.00,0.50,0.00}{##1}}}
\expandafter\def\csname PY@tok@kd\endcsname{\let\PY@bf=\textbf\def\PY@tc##1{\textcolor[rgb]{0.00,0.50,0.00}{##1}}}
\expandafter\def\csname PY@tok@kn\endcsname{\let\PY@bf=\textbf\def\PY@tc##1{\textcolor[rgb]{0.00,0.50,0.00}{##1}}}
\expandafter\def\csname PY@tok@kr\endcsname{\let\PY@bf=\textbf\def\PY@tc##1{\textcolor[rgb]{0.00,0.50,0.00}{##1}}}
\expandafter\def\csname PY@tok@bp\endcsname{\def\PY@tc##1{\textcolor[rgb]{0.00,0.50,0.00}{##1}}}
\expandafter\def\csname PY@tok@fm\endcsname{\def\PY@tc##1{\textcolor[rgb]{0.00,0.00,1.00}{##1}}}
\expandafter\def\csname PY@tok@vc\endcsname{\def\PY@tc##1{\textcolor[rgb]{0.10,0.09,0.49}{##1}}}
\expandafter\def\csname PY@tok@vg\endcsname{\def\PY@tc##1{\textcolor[rgb]{0.10,0.09,0.49}{##1}}}
\expandafter\def\csname PY@tok@vi\endcsname{\def\PY@tc##1{\textcolor[rgb]{0.10,0.09,0.49}{##1}}}
\expandafter\def\csname PY@tok@vm\endcsname{\def\PY@tc##1{\textcolor[rgb]{0.10,0.09,0.49}{##1}}}
\expandafter\def\csname PY@tok@sa\endcsname{\def\PY@tc##1{\textcolor[rgb]{0.73,0.13,0.13}{##1}}}
\expandafter\def\csname PY@tok@sb\endcsname{\def\PY@tc##1{\textcolor[rgb]{0.73,0.13,0.13}{##1}}}
\expandafter\def\csname PY@tok@sc\endcsname{\def\PY@tc##1{\textcolor[rgb]{0.73,0.13,0.13}{##1}}}
\expandafter\def\csname PY@tok@dl\endcsname{\def\PY@tc##1{\textcolor[rgb]{0.73,0.13,0.13}{##1}}}
\expandafter\def\csname PY@tok@s2\endcsname{\def\PY@tc##1{\textcolor[rgb]{0.73,0.13,0.13}{##1}}}
\expandafter\def\csname PY@tok@sh\endcsname{\def\PY@tc##1{\textcolor[rgb]{0.73,0.13,0.13}{##1}}}
\expandafter\def\csname PY@tok@s1\endcsname{\def\PY@tc##1{\textcolor[rgb]{0.73,0.13,0.13}{##1}}}
\expandafter\def\csname PY@tok@mb\endcsname{\def\PY@tc##1{\textcolor[rgb]{0.40,0.40,0.40}{##1}}}
\expandafter\def\csname PY@tok@mf\endcsname{\def\PY@tc##1{\textcolor[rgb]{0.40,0.40,0.40}{##1}}}
\expandafter\def\csname PY@tok@mh\endcsname{\def\PY@tc##1{\textcolor[rgb]{0.40,0.40,0.40}{##1}}}
\expandafter\def\csname PY@tok@mi\endcsname{\def\PY@tc##1{\textcolor[rgb]{0.40,0.40,0.40}{##1}}}
\expandafter\def\csname PY@tok@il\endcsname{\def\PY@tc##1{\textcolor[rgb]{0.40,0.40,0.40}{##1}}}
\expandafter\def\csname PY@tok@mo\endcsname{\def\PY@tc##1{\textcolor[rgb]{0.40,0.40,0.40}{##1}}}
\expandafter\def\csname PY@tok@ch\endcsname{\let\PY@it=\textit\def\PY@tc##1{\textcolor[rgb]{0.25,0.50,0.50}{##1}}}
\expandafter\def\csname PY@tok@cm\endcsname{\let\PY@it=\textit\def\PY@tc##1{\textcolor[rgb]{0.25,0.50,0.50}{##1}}}
\expandafter\def\csname PY@tok@cpf\endcsname{\let\PY@it=\textit\def\PY@tc##1{\textcolor[rgb]{0.25,0.50,0.50}{##1}}}
\expandafter\def\csname PY@tok@c1\endcsname{\let\PY@it=\textit\def\PY@tc##1{\textcolor[rgb]{0.25,0.50,0.50}{##1}}}
\expandafter\def\csname PY@tok@cs\endcsname{\let\PY@it=\textit\def\PY@tc##1{\textcolor[rgb]{0.25,0.50,0.50}{##1}}}

\def\PYZbs{\char`\\}
\def\PYZus{\char`\_}
\def\PYZob{\char`\{}
\def\PYZcb{\char`\}}
\def\PYZca{\char`\^}
\def\PYZam{\char`\&}
\def\PYZlt{\char`\<}
\def\PYZgt{\char`\>}
\def\PYZsh{\char`\#}
\def\PYZpc{\char`\%}
\def\PYZdl{\char`\$}
\def\PYZhy{\char`\-}
\def\PYZsq{\char`\'}
\def\PYZdq{\char`\"}
\def\PYZti{\char`\~}
% for compatibility with earlier versions
\def\PYZat{@}
\def\PYZlb{[}
\def\PYZrb{]}
\makeatother


    % Exact colors from NB
    \definecolor{incolor}{rgb}{0.0, 0.0, 0.5}
    \definecolor{outcolor}{rgb}{0.545, 0.0, 0.0}



    
    % Prevent overflowing lines due to hard-to-break entities
    \sloppy 
    % Setup hyperref package
    \hypersetup{
      breaklinks=true,  % so long urls are correctly broken across lines
      colorlinks=true,
      urlcolor=urlcolor,
      linkcolor=linkcolor,
      citecolor=citecolor,
      }
    % Slightly bigger margins than the latex defaults
    
    \geometry{verbose,tmargin=1in,bmargin=1in,lmargin=1in,rmargin=1in}
     

    \begin{document}
    
    
    \maketitle
    
    



    \hypertarget{domain-background}{%
\section{Domain Background}\label{domain-background}}

The chosen project for the Capstone Project is the
\href{https://www.kaggle.com/c/nyc-taxi-trip-duration}{New York City
Taxi Duration Trip} competition from
\href{https://en.wikipedia.org/wiki/Kaggle}{Kaggle}. The challenge is to
build a model that predicts the total ride duration of taxi trips in New
York City.

Solving problems alike is not a new challenge
{[}\href{https://www.aaai.org/Papers/KDD/1998/KDD98-037.pdf}{1},\href{https://ieeexplore.ieee.org/document/518845}{2}{]},
however, it was not ultil recently that the amount of data has increased
enough, and become more precise with the rise hightech smartphones and
GPS use and real-time tracking, that solutions were made possible.

    \hypertarget{problem-statement}{%
\subsubsection{2. Problem Statement}\label{problem-statement}}

By considering historical information such as pick up and drop off date,
hour and geo-localization, and trip duration regarding taxi trips, the
objective is to predict the duration of each taxi trip given in a
specific test set, which does not present neither drop off time, nor
trip duration. The prediction will be performed through a supervised
learning regressor to be defined given the data structure.

    \hypertarget{datasets-and-inputs}{%
\subsubsection{3. Datasets and Inputs}\label{datasets-and-inputs}}

The challenge provides two
\href{https://www.kaggle.com/c/nyc-taxi-trip-duration/data}{data sets}.
* train.csv - the training set, which contains 1.458.644 observations *
test.csv - the testing set, which contains 625.134 observations

The training set contains eleven features:

\begin{itemize}
\tightlist
\item
  \texttt{id} - a unique identifier for each trip
\item
  \texttt{vendor\_id} - a code indicating the provider associated with
  the trip record, \emph{categorical}
\item
  \texttt{pickup\_datetime} - date and time when the meter was engaged,
  \emph{categorical}
\item
  \texttt{dropoff\_datetime} - date and time when the meter was
  disengaged
\item
  \texttt{passenger\_count} - the number of passengers in the vehicle
  (driver entered value)
\item
  \texttt{pickup\_longitude} - the longitude where the meter was engaged
\item
  \texttt{pickup\_latitude} - the latitude where the meter was engaged
\item
  \texttt{dropoff\_longitude} - the longitude where the meter was
  disengaged
\item
  \texttt{dropoff\_latitude} - the latitude where the meter was
  disengaged
\item
  \texttt{store\_and\_fwd\_flag} - This flag indicates whether the trip
  record was held in vehicle memory before sending to the vendor because
  the vehicle did not have a connection to the server - Y=store and
  forward; N=not a store and forward trip
\item
  \texttt{trip\_duration} - duration of the trip in seconds
\end{itemize}

The testing set doesn't have neither the \texttt{trip\_duration}, target
variable, nor the \texttt{dropoff\_datetime}. So,the model should be
trained on the training set, and predict the \texttt{trip\_duration} for
the testing set observations. The predictions should be
\href{https://www.kaggle.com/c/nyc-taxi-trip-duration/submit}{submitted
to the Kaggle's challenge} for calculating the final score.

The variables \texttt{pickup\_latitude}, \texttt{pickup\_longitude},
\texttt{dropoff\_latitude} and \texttt{dropoff\_longitude} will be used
to calculate the trip distance between the pickup and dropoff locations.
Also, they are important and will be analysed because where in NY the
each trip starts and ends influences greatly the velocity in which the
taxi can travel.

The variables \texttt{pickup\_datetime} are important, and also will be
analysed profoundly, because the moment on time in which the trips occur
also influences greatly the the taxi velocity.

The remaining variables will be analysed in order to conclude about
their influence on the target variable.

The data sets provided by the Kaggle challenge were made avaiable in
\href{https://cloud.google.com/bigquery/public-data/nyc-tlc-trips}{Big
Query on Google Cloud Platform}, originally published by
\href{https://www1.nyc.gov/site/tlc/about/tlc-trip-record-data.page}{NYC
Taxi and Limousine Commission (TLC)}.

    \hypertarget{solution-statement}{%
\section{Solution Statement}\label{solution-statement}}

The solution will be provided by a regression method. The optimization
will be performed after a preprocessing of the data and through
minimization of the root mean square logarithm error (RMSLE), cross
validation for detecting over and/or underfitting, and grid-search for
fine tunning hyperparameters.

    \hypertarget{benchmark-model-and-result}{%
\section{Benchmark Model and Result}\label{benchmark-model-and-result}}

The benchmark model will be the raw linear regression as it is well
known and simple.

The benchmark result will be the
\href{https://www.kaggle.com/c/nyc-taxi-trip-duration/leaderboard}{winning}
score of the Kaggle's challenge when the winning team submitted the
provided testing set. The challenge's metric is the RMSLE, and the
winning score is \texttt{0.28976}.

    \hypertarget{evaluation-metrics}{%
\section{Evaluation Metrics}\label{evaluation-metrics}}

The evaluation metric is the one from the Kaggle's challenge,
\href{https://www.quora.com/What-is-the-difference-between-an-RMSE-and-RMSLE-logarithmic-error-and-does-a-high-RMSE-imply-low-RMSLE}{RMSLE},
in order to make the score comparable to the benchmark result. The
metric is quite similar to the
\href{https://en.wikipedia.org/wiki/Root-mean-square_deviation}{root
mean squared error}, however, considers the squared difference of the
logarithm values of prediction (\(\hat{y}\)) and true values(\(y\)).

\[\text{RMSLE}(y,\hat{y}) = \sqrt{\frac{1}{N}\sum_k{\text{log}(1+y_k)-\text{log}(1+\hat{y}_k)}}\]

    \hypertarget{project-design}{%
\section{Project Design}\label{project-design}}

The project will be segregated in four major parts: data preprocessing,
model selection, benchmark comparison and conclusion.

\hypertarget{data-preprocessing}{%
\subsection{Data Preprocessing}\label{data-preprocessing}}

Data preprocessing will be segregated into two sections: Generals and
Specifics.

\hypertarget{generals}{%
\subsubsection{Generals}\label{generals}}

In this section, data quality will be considered and performed.

\begin{itemize}
\tightlist
\item
  Are there any missing values?
\item
  The features are formatted on a convenient way?
\item
  Is there any unnecessary information?
\end{itemize}

Along with that, from the features \texttt{pickup\_latitude},
\texttt{pickup\_longitude}, \texttt{dropoff\_latitude} and
\texttt{dropoff\_longitude}, the feature \texttt{distance} will be
calculated.

\hypertarget{specifics}{%
\subsubsection{Specifics}\label{specifics}}

In this section, the data will be deeply explored with the objective of
understanding the features influences, finding transformations and
detecting and dropping outliers in order to make the training and
prediction better.

\textbf{Continuous data} First step will look for skewness of the
continuous data and propose transformations that will be Logarithmic or
Box-Cox and, finally. A probability plot against a normal distribution
will help the decision in which transformation to perform.

Second step, take a good look at the data scatter plotting continuous
features against each other and looking at their distributions to search
for outliers, that will be studied further to decide wheter they should
be dropped or not.

Finally, the data will be normalized to the interval \([0,1]\).

\textbf{Categorical data} The influence of categorical data will be
studied separetely. First, datetime related will be segregated into
\texttt{month}, \texttt{day}, \texttt{hour}, \texttt{day\ of\ the\ week}
and binary variables that flag if it's a holiday or not, in orther to
search for periodical/special behaviour.Second, the other categorical
features such as \texttt{passenger\_count} and \texttt{vendor\_id}. They
will be transform to dummy variables through \emph{one-hot-encoding}.
For these analysis, the velocity, \texttt{distance} over
\texttt{trip\_duration}, will be taken into account as the trip duration
may depend on traffic, that depends greatly on date and time.

\hypertarget{model-selection}{%
\subsection{Model Selection}\label{model-selection}}

This parte will train the models and consider the RMSLE scoring for
choosing the best one. As the testing set provided doesn't have the
target feature, \texttt{trip\_duration}, the training set will be
splitted into training and testing set for the model selection.

\hypertarget{initial-model-evaluation}{%
\subsubsection{Initial Model
Evaluation}\label{initial-model-evaluation}}

For the initial model evaluation, a set of different regressors will be
trained in different data set sizes, using cross validation, in order to
verify underfitting and overfitting between regressors. No tunning of
hyperparameters will be performed.

As there will be a lot of dummy binary variables after the
\emph{one-hot-encoding} transformation, ensemble methods will be a good
option for the final model. In this step, will be trained:

\begin{itemize}
\tightlist
\item
  Simple Linear Regression (benchmark model)
\item
  Decision Trees
\item
  Rain Forest
\item
  Bagging
\item
  Gradient Boosting
\item
  Extreme Boosting
\end{itemize}

\hypertarget{model-tunning}{%
\subsubsection{Model Tunning}\label{model-tunning}}

The best two models from the previous step will have their
hyperparameters tunned through gridsearch. The results of the best set
of parameters for each model will be compared and the final model
decided.

\hypertarget{final-validation}{%
\subsubsection{Final Validation}\label{final-validation}}

Once chosen the best model, it must pass through a set of final
validations in order to check its robustness.

\hypertarget{benchmark-comparison}{%
\subsection{Benchmark Comparison}\label{benchmark-comparison}}

Once the model has been selected and trained on the whole provided
training set, the predictions over the provided testint set will be
submitted to the private
\href{https://www.kaggle.com/c/nyc-taxi-trip-duration/leaderboard}{leaderboard}
on Kaggle. The result will be compared with the bechmark model, Linear
Regression, and the benchmark result, which is \texttt{0.28976}.

\hypertarget{conclusion}{%
\subsection{Conclusion}\label{conclusion}}

The last section it is the conclusion with a reflextion about the data,
the model and a proposal possible improvements.


    % Add a bibliography block to the postdoc
    
    
    
    \end{document}
